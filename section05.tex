\section{CONCLUSÃO}

	Com adoção da TRI pelo ENEM, a TRI ganhou notoriedade no cenário acadêmico. com o avanço computacional. A utilização da TRI em avaliações das escolas e não apenas 
	\paragraph{}
	A TRI tem uma complexidade matemática que não é entendia por todos. Então para que esta metologia seja usada pelas escolas para uma avaliação qualitativa, é necessario grande empenho dos gestores, para qualificação de seus professores.
	Apesar dos avanços da área computacional os softwares para TRI ainda não cobrem todos os modelos da TRI, tendo o pesquisador que criar seus próprios programas para estimação das habilidades e parâmetros dos itens.
	Como a TRI entrou no Brasil com o objetivo de aprimorar as avaliações educacionais, e considerando a sua aplicação atual no ENEM, não é de se admirar que a maior parte das aplicações tem sido realizada na área de avaliação educacional.
\cite{Dalton}
\newpage