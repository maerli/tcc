\section{CONCLUSÃO}
    \par
    	Com adoção da TRI pelo ENEM, a TRI ganhou notoriedade no cenário acadêmico. Com o avanço computacional e a utilização da TRI em avaliações nacionais e internacionais esta tende a ganha notoriedade no campo acadêmico e educacional.
	\par
    	A TRI tem uma complexidade matemática que não é entendia por todos. Então para que esta metologia seja usada pelas escolas para uma avaliação qualitativa, é necessário grande empenho dos gestores, para qualificação de seus professores.
    \par
	    Apesar dos avanços da área computacional os softwares para TRI ainda não cobrem todos os modelos da TRI, tendo o pesquisador que criar seus próprios programas para estimação das habilidades e parâmetros dos itens. ATRI entrou no Brasil com o objetivo de aprimorar as avaliações educacionais, e considerando a sua aplicação atual no ENEM, não é de se admirar que a maior parte das aplicações tem sido realizada na área de avaliação educacional.
\newpage