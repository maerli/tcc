\section{INTRODUÇÃO}
	\paragraph{}
	    \cite{MOREIRAJUNIOR} No início da colonização Brasileira, com a chegada dos portugueses juntamente com a (\textcite{OLIVEIRA}) Igreja Católica e também com a chegada da Ordem dos Jesuítas no ano de 1549. Os Jesuítas que acompanharam as grandes navegações, que tiveram atuação no Brasil de 1549 a 1759, estes tinham o objetivo de ensinar, catequizar e evangelizar a população nativa e para verificar os resultados de seus trabalhos aplicavam uma espécie de avaliação informal, chamadas provas orais, isto é, com o objetivo de analisar os ensinamentos dados a esta população de nativos sobre o Deus católico e a submissão à coroa portuguesa, verificar a aculturação e a conversão dos nativos, estes eram avaliados a partir da exposição de seus pensamentos pela oralidade, e escrita, em algumas situações. sendo esta uma forma de avaliação que apesar de válido, mostra-se totalmente dependente dos envolvidos.
	\paragraph{}
    	Nas décadas de 30 e 40, a partir da organização do Manifesto dos Pioneiros da Escola Nova, o processo de ensino e de aprendizagem é marca pela visão humanista do homem em relação ao homem tradicional, ou seja, há um equilíbrio no que diz respeito a concepção de homem, porém esse período não teve mudanças com importância significativa no que diz respeito ao modo de avaliar o aprendizado dos estudantes. Avaliar é parte fundamental do aspecto humano. todos nós somos avaliados de uma forma ou de outra, de forma voluntária quando postamos conteúdos nas redes sociais e o número de curtidas indicaram a aceitação do mesmo perante os amigos. Estes eventos ocorrem também no mundo acadêmico professores,alunos e instituições são avaliadas. Alunos são avaliados pelos professores a fim de verificarem a aprendizagem dos mesmos. Professores são avaliados a fim de avaliar o progresso diante do conteúdo a ser ministrado. As instituições são avaliadas por instâncias superiores como o MEC(Ministério da Educação). Todo este processo nos levar a acreditar que métodos  que tornem avaliações mais precisas devem ser implementados a TRI tem esse objetivo e o cumpre em partes. Daí no âmbito da educação avaliar faz parte de todo o processo de ensino. Segundo as leis brasileiras a avaliação tem esse fim.
	\par
	    \textcite{LDB} O Art. 9º, inciso VI da Lei de Diretrizes e Bases da Educação (LEI Nº 9.394, DE 20 DE DEZEMBRO DE 1996) versa sobre o objetivo da avaliação escolar, cabendo a União
		"VI - Assegurar processo nacional de avaliação do rendimento escolar no ensino fundamental, médio e superior, em colaboração com os sistemas de ensino, objetivando a definição de prioridades e a melhoria da qualidade do ensino."(BRASIl,1996,pag. 27833)
	\par
	    O processo avaliativo visa a melhoria e qualidade do ensino, então instrumentos que possibilitem uma melhor checagem dos dados também faz-se necessário.Por isso a TRI, supre essa carência de fala de instrumentos. 
	\par
	    Mediante as concepções que os alguns autores têm sobre avaliação podemos estabelecer critérios para avaliar, segundo Luckesi, "a avaliação é uma apreciação qualitativa sobre dados relevantes do processo de ensino e aprendizagem que auxilia o professor a tomar decisões sobre o seu trabalho.” Isso sugere que os dados quantitativos não são suficientes para avaliar há a necessidade que a avaliação seja também qualitativa, por tanto os instrumentos de medida devem prover os dados quantitativos e também uma forma de avaliar qualitativamente. Ainda segundo Luckesi (1978) a avaliação é definida como um julgamento de valor sobre manifestações relevantes da realidade, tendo em vista uma tomada de decisão.
	\par
	    Diante deste ponto de vista sobre avaliação. destaca-se os três principais tipos de avaliação.
	\subsection{TIPOS DE AVALIAÇÃO}
	\paragraph{}
	\subsection{avaliação diagnóstica}
	\paragraph{}
	    Tem como pressuposto básico detectar ou fazer uma verificação dos conteúdos e conhecimento do aluno. E a partir dos dados desse diagnóstico realizar o planejamento de ações que supram as necessidades e atinja os objetivos propostos. Com isso se utiliza a avaliação de aprendizagem como suporte para o planejamento de ensino. Sendo este tipo de avaliação aplicado no início do processo de ensino-aprendizagem.
	\subsection{avaliação formativa}
	\paragraph{}
	    Tem o objetivo verificar se o conteúdo proposto pelo professor no seu planejamento em relação aos conteúdos estão sendo atingidos durante todo o processo de ensino aprendizagem do aluno passo a passo. Com isso é possível aplicar a recuperação paralela, onde os alunos resgatam os conceitos revisando-os ao longo do caminho e evoluindo cada um no seu ritmo.
	\par
	    Essa intervenção e postura do professor como mediador tira de cena aquela prática de classificar o aluno com uma nota. Não se tem mais a visão da avaliação no resultado do teste e sim no potencial de desenvolvimento do aluno. O professor como mediador refletirá sobre o processo e tomar decisões para re-planejar suas ações para intervir e adequar suas práticas em sala de aula com o objetivo do aluno aprender e não simplesmente melhorar sua nota.
	\subsection{avaliação somativa}
	\paragraph{}
	    Tipo de avaliação que ocorre ao final da instrução com a finalidade de verificar o que o aluno efetivamente aprendeu. Inclui conteúdos mais relevantes e os objetivos mais amplos do período de instrução; visa à atribuição de notas; fornece \textit{feedback} ao aluno (informa-o quanto ao nível de aprendizagem alcançado), se este for o objetivo central da avaliação formativa; e presta-se à comparação de resultados obtidos com diferentes alunos, métodos e materiais de ensino. Foi assim classificada por Benjamin Bloom e seus colaboradores, cujos estudos apontam para outros dois tipos de avaliação: a formativa e a diagnóstica.\textcite{Takuno}
	\par
	    Tem o objetivo de atribuir notas e conceitos para o aluno ser promovido ou não de uma classe para outra, ou de um curso para outro, normalmente realizada durante o bimestre ou semestre.
		Segundo Luckesi, O professor deve aprender a avaliar, os professores devem aprender a avaliar, ou seja, professores são parte do processo avaliativo.
	    A TRI pode ser aplicada a cada uma destes três principais formas de avaliação.
	
	
	\newpage