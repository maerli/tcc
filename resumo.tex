\begin{center}
	    \textbf{RESUMO}
	\end{center}
	
	Desde os primórdios da sociedade humana moderna tem-se buscado avaliar as atividades humanas e suas respectivas habilidades, entretanto a forma que esta avaliação se dava era muito subjetiva, até que em meados do século 20, especialistas como matemáticos e psicomotristas criaram metologias capazes de medir grandezas próprias da natureza humana como habilidade ou proficiência em uma determinada área do conhecimento, com a criação da TCT evolui-se na medida de traços latentes, porém esta teoria tinha algumas limitações, a TRI veio para suprir estas carências, criando assim uma teoria capaz de medir com mais precisão os traços latentes, a TRI tem como característica básica a calibragem dos itens,ou seja a obtenção por meio de testes de três parâmetros: dificuldade, discriminação e acerto ao acaso. Devido a adoção o Enem da TRI,2009, no Brasil a TRI ganhou mais destaque, entretanto a complexidade matemática tem desencorajado professores na adoção desta metodologia para o uso em suas avaliações.
	\par
	\noindent Palavras Chaves: TRI, ENEM, teoria da medida.
	\newpage